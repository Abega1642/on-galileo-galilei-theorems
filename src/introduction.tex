\section*{Introduction}
\quad	Galileo Galilei (1564–1642) is widely regarded as one of the greatest minds in the history of science. His contributions to physics, astronomy, and mathematics laid the groundwork for modern scientific thought. Born in Pisa, Italy, Galileo’s early studies in medicine quickly shifted toward mathematics, where his true genius emerged. He revolutionized observational astronomy with his improvements to the telescope, famously discovering the moons of Jupiter and supporting the heliocentric model of the solar system. However, his challenges to Aristotelian physics were perhaps just as groundbreaking, setting the stage for classical mechanics. Despite facing opposition from the Catholic Church and even enduring house arrest in his later years, Galileo’s legacy remains foundational to modern science.\\

\quad	One of his most important works, Dialogues on the Two New Sciences (1638), was written during his house arrest and is often considered his final major contribution. This book is structured as a conversation between three fictional characters—Salviati, Sagredo, and Simplicio—who debate and discuss the principles of mechanics and motion. The "two new sciences" in question are the science of material strength and the science of motion, both of which became essential foundations for classical physics. The book masterfully blends rigorous mathematical reasoning with experimental observations, marking one of the earliest examples of a true scientific method.\\

\quad	In particular, the "Third Day" of the Dialogues focuses on Galileo’s mathematical treatment of motion. Here, he introduces key concepts such as the uniform acceleration of falling bodies, the parabolic trajectory of projectiles, and the law of free fall. He formulates his discoveries using geometric reasoning, constructing theorems, corollaries, lemmas, and propositions in the style of classical Greek mathematics, reminiscent of Euclid and Archimedes. While this method was rigorous for its time, it is far removed from the symbolic notation and modern formalism used in contemporary mathematics and physics. \\

\quad	The purpose of this article is to bridge this gap. Galileo’s theorems, corollaries, lemmas, and propositions will be reformulated using modern mathematical notation and expressions, making his results more accessible to today’s mathematicians and physicists. By translating his arguments into the language of calculus and modern algebra, we will highlight the enduring relevance of his insights and provide a clearer understanding of his revolutionary contributions to the science of motion.

