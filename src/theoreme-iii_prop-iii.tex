\section{On the theorem III - Proposition III}
This next section, we will deal with the \textit{Theorem III, Proposition III} of the Third Day which taks about the natural accelerated motion.
\begin{theorem}
If one and the same body, starting from rest, falls along  an inclined plane and also a vertical, each having the same height, the times of descent will be to each oher as the lengths of the inclined plane and the vertical.
\end{theorem}

This theorem states that if a body, starting from rest, descends along both an inclined plane and a vertical drop of the same height, the time taken to reach the bottom in each case will be proportional to the respective distances traveled.\\

The following demonstration lies on the previous theorem.

\begin{proof}
According to the last theorem, equation (2), we have $v_1 = v_2$.\\

If we come back to the definition of speed, the speed of a body traveling a distance $d$ during a precise time $t$ is given by :

\[v = \frac{d}{t} \]

Thus, we have :

\[v_1 = v_2 = \frac{AB}{t_1} = \frac{AE}{t_2} \]
\[\Longrightarrow	\quad	\frac{AB}{AE} = \frac{t_1}{t_2} \]

Now, we can notice that the last theorem is true $\forall$ $\beta$ and $\alpha$.\\

So, taking $\beta = \frac{\pi}{2}$, therefore, the theorem is prooved.
\end{proof}
