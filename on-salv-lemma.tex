\section{On the Salviati lemma}
\qquad On the \textit{Third day, Change of position, [De Motu Locali]}, Galileo deals with the notions of motions, acceleration, and more precisely, he fonded the \textit{Dynamical mechanics}.\\

\quad One of the important theorem he stated in this \textit{Third day} is the one that \textit{Salviati} states after the \textit{Scholium} of the \textit{Corollary II}. on page 183-184\\

\begin{theorem}
If a body falls freely along smooth planes inclined at any angle whatsoever, but of the same height, the speeds with which it reaches the bottomare the same.
\end{theorem}

\quad Galileo's demonstration of this theorem is very old and might be difficult to apprehend easily, i.e, the expressions and the notations are very old. But hereafter, I'm going to proove that theorem in a modern notation and modern expression which will make the demonstration easy to understand and to apprehend easily.\\	\\

\underline{\textit{Proof.}}
Let \textit{ABC} be a rectangle triangle.\\

\tikz	
\draw	(0,-2)	--	(3,-2)	-- (3,1)		--	(1.3, -2)	--	(3,1)	--	(0,-2);


Let us lay down :
\[AC = h; \]
\[ AB = x_1;	\]
\[AE = x_2;	\] \\

As we know :
\[
sin(\alpha) = \frac{h}{x_1};
\]
\[
sin(\beta) = \frac{h}{x_2};
\]

That gives us :
\[
h = x_1sin(\alpha);
\]
\[
h = x_2sin(\beta);
\]
As we know $  x_1 = \frac{1}{2}\cdot a_1\cdot t_1^{2}	$ where $ a_1 $ is the acceleration on the plane $ AB $, and $ t_1 $ the time required to travers it; and $ x_2 = \frac{1}{2}\cdot a_2\cdot t_2^{2} $. where $ a_2 $ the acceleration on the plane $ AE $, and $ t_2 $ the time required to travers that plane.
